\documentclass{article}

\usepackage{amsmath,amssymb,cancel}
\usepackage{hyperref}
\usepackage{qtree}
\newcommand{\X}{\mathcal{X}}


\title{SS 202A: }
\author{George Vega Yon\and Nicholas Adams-Cohen}

% Some useful macros
\def\reals{\mathbb{R}}
\def\noR{\cancel{R}}

\begin{document}
\maketitle


\section*{Lecture 1: Condosect Paradox. Sept 30th, 2014}

\subsection*{Example 1: Social Choice Theory}
Ben, Matt, and Alex are on a search comittee to hire the next faculty member. They have the following preferences: 

\begin{table}[h]
\centering
\begin{tabular}{llll}
                                         & Alex                            & Matt                            & Ben                             \\ \cline{2-4} 
\multicolumn{1}{l|}{Political Scientist} & \multicolumn{1}{l|}{\textbf{1}} & \multicolumn{1}{l|}{\textbf{3}} & \multicolumn{1}{l|}{\textbf{2}} \\ \cline{2-4} 
\multicolumn{1}{l|}{Theorist}            & \multicolumn{1}{l|}{\textbf{2}} & \multicolumn{1}{l|}{\textbf{1}} & \multicolumn{1}{l|}{\textbf{3}} \\ \cline{2-4} 
\multicolumn{1}{l|}{Econometrician}      & \multicolumn{1}{l|}{\textbf{3}} & \multicolumn{1}{l|}{\textbf{2}} & \multicolumn{1}{l|}{\textbf{1}} \\ \cline{2-4} 
\end{tabular}
\end{table}

\textbf{Note:} Every actor is rational in that the have a clear rank order. But could a collective decision be reached?

Under \underline{Majority Rule}:
\begin{itemize}

\item (Alex \& Ben) P $>$ T
\item (Matt \& Alex) T $>$ M
\item (Ben \& Matt) M $>$ P
\end{itemize}

We define \underline{social ranking} as $xPy$ if and only if a majority prefer x to y. 
\emph{\textbf{But:}}

\begin{align*}
P &> T \\
T&>M\\
M&>P
\end{align*}

Which violates transitivity.

\subsection*{Example 2: Game Theory}

What if you create some institution with some rules and structure that will allow us to determine what people will do? 

\textbf{\underline{Binary Agenda Setting:}}

We have a sequence of votes of $x$ vs. $y$ which obeys the following rules:

\begin{itemize}

\item Finite.

\item The winner of round $t$ is one of the alternatives in round $t+1$.

\item The winner of the final round prevails.
\end{itemize}
\textbf{Tree:} If Alex was the agenda-setter, begin with Metrics vs. Political Science. If Metrics prevails, in the second round the vote will be Metrics vs. Theory, and Theory will win the vote, so a vote for Metrics in the first round is \emph{in actuality} a vote for Theory. If Political Science prevails, it will be a vote of Political Science vs. Theory, and Political Science will win, so a vote for Political Science in the first round is actually a vote for Political Science.

Thus, by setting the agenda in sequential voting, Metrics vs. Political Science is really a vote for Theory vs. Political Science and Political Science prevails.  %Insert Tree if time

\emph{*Takeaway:} By adding structure, we can form a point prediction. By manipulating what is voted for in the first round, you can determine the victor in the second round.

However, the agenda-setting power is limited. Consider the following preferences:

\begin{table}[h]
\centering
\begin{tabular}{llll}
                                         & Alex                            & Matt                            & Ben                             \\ \cline{2-4} 
\multicolumn{1}{l|}{Political Scientist} & \multicolumn{1}{l|}{\textbf{1}} & \multicolumn{1}{l|}{\textbf{3}} & \multicolumn{1}{l|}{\textbf{3}} \\ \cline{2-4} 
\multicolumn{1}{l|}{Theorist}            & \multicolumn{1}{l|}{\textbf{2}} & \multicolumn{1}{l|}{\textbf{1}} & \multicolumn{1}{l|}{\textbf{2}} \\ \cline{2-4} 
\multicolumn{1}{l|}{Econometrician}      & \multicolumn{1}{l|}{\textbf{3}} & \multicolumn{1}{l|}{\textbf{2}} & \multicolumn{1}{l|}{\textbf{1}} \\ \cline{2-4} 
\end{tabular}
\end{table}

Now, though Alex still prefers Political Science, there is nothing he can do as agenda setter to get this outcome. 

\subsection*{Example 3: How to choose an Agenda Setter?}

In the previous example, Alex was ``given'' as the agenda setter \dots but how was he choosen? 
Common approach is to make this a 2-stage process: 
\begin{enumerate}

\item How to choose an agenda-setter?
\item How will everyone vote? 
\end{enumerate}

We can write out everyone's preferences for an agenda-setter\dots but we quickly realize that preferences over an agenda-setter (who have the ability to determine the next faculty member) become peoples preferences over the next faculty member. That is:

\begin{table}[h]
\centering
\begin{tabular}{llll}
                                         & Alex                            & Matt                            & Ben                             \\ \cline{2-4} 
\multicolumn{1}{l|}{Alex A.S. = (P)} & \multicolumn{1}{l|}{\textbf{1}} & \multicolumn{1}{l|}{\textbf{3}} & \multicolumn{1}{l|}{\textbf{2}} \\ \cline{2-4} 
\multicolumn{1}{l|}{Matt A.S. = (T)}            & \multicolumn{1}{l|}{\textbf{2}} & \multicolumn{1}{l|}{\textbf{1}} & \multicolumn{1}{l|}{\textbf{3}} \\ \cline{2-4} 
\multicolumn{1}{l|}{Ben A.S. = (M)}      & \multicolumn{1}{l|}{\textbf{3}} & \multicolumn{1}{l|}{\textbf{2}} & \multicolumn{1}{l|}{\textbf{1}} \\ \cline{2-4} 
\end{tabular}
\end{table}

So choosing an institutional structure ``inherets'' the previous preferences. So we are going to need \emph{more} knowledge/ assumptions about the institutional structure.

\emph{*Takeaway:} Math alone does not give us the answer.

\subsection*{Three Themes of the Course}
\begin{enumerate}

\item \underline{Will of Society} is often a meaningless concept.
\item Adding \underline{institutions} can help\dots
\item But only to a point. \underline{Assumptions} matter, and you need them.
\end{enumerate}%test

\section*{Lecture 2}


When will underlying preferences lead to rationaliable choices?

$X \equiv \text{ set of alternatives } |X| \geq 3$. 

$\X \equiv$ set of all subsets $(X\setminus \emptyset)$ or $(2^\X \setminus \emptyset)$ (powerset). 

$S\subset X \Leftrightarrow \in \X$.

\bigskip

\underline{Derive choices from Weak Preference Relation}

M(R,S) = $\{x\in S:\forall y \in S, xRy\}$)\footnote{R is a preference, $S \in \X$.}

Minimal expected condition ``rationality'' is that the underlying preference M(R,S) $\neq \emptyset (\forall S \in \X)$

\subsection*{Three {roperties of the W.P. Relation}}


\begin{enumerate}

\item R transitive iff $\forall x,y,z \in X, xRy, yRz \Rightarrow xRz$ ``weak orders.''
\item R is quasi-transitive $\forall x,y,z \in X, xPy, yPz \Rightarrow xPz.$
\item Acylicity: $\forall x,y,z,\dots w,v$ then $xPyPz \dots wPv \Rightarrow xRv$
\end{enumerate}

\underline{Theorem 1.1}
Acylity is necessary and sufficient for maximal sets to be non empty $\forall \; s\in \X$

\textbf{\emph{Proof:}}

\underline{Necessity} $(\Leftarrow)$ (by contradiction.)

We know that the maximal set is not empty. 
Assume $xPyPz\dots uPv$ but $\neg(xRv)$.

So $\neg(vRu) \dots \neg(zRy),\neg(yRx)$ by the definition of P, and $\neg(xRv)$ by assumption.

But then M(R,$ \{ \cdot,\}$) is empty. Which is a contradiction.

\underline{Sufficiency}
\begin{enumerate}
\item If acylity is satisified, then pick $S \in \X$. 
\item Pick an arbitrary element $x_1 \in S$.
\item Compare $x_1$ with $S\setminus\{x_1\} $. If $x_1 R x_i (\forall i)$, we are done.
\item If $\neg (x_1 R y) \; \exists x_2 \in S$ such that $x_2Px_1$.
\item If $x_2 R x_i (\forall i)$, again we are done. 
\item If not you can follow the above steps and eventually find an $x' R x_i (\forall i)$. Hence, the set is acylic.  
\end{enumerate}


---


Choice function is a \underline{map} $C: \X \to \X$. That is for every sebset, $C$ gives you a subset of the subset. Ex $C(\{x,y,z\}) = \{x,y\} $.

\bigskip

Choice function satisfies \textbf{transitive rationalizability} iff $\exists$ a binary relation s.t. $S\in \X$, $C(S)=M(R,S)$ and moreover R is transitive.

\bigskip

\underline{Conditions:}


\begin{itemize}

\item Condition $\alpha$
\item Condition $\beta$
\item Weak Axiom of Revealed Preferences
\end{itemize}

-Does sensible things when you add and subtract alternatives.

\begin{itemize}

\item 
Condition $\alpha$: condition consistency property.

\[
\forall S,T; S\subset T \;\;\;C(T) \cap S \subset C(S)
\]

If I pick a bigger set and shrink it to a smaller set with choice still there, you will still pick it in a smaller set.

\item 
Condition $\beta$: Expansion-consitency property.

$S \subset T  \;\;\; C(S)\cap C(T) = \emptyset \Rightarrow C(S) \subset C(T) $


\item WARP

If $x\in C(S)$ and $y \not \in C(S)$ and $y \in C(T) \Rightarrow x \not \in T$ 

\end{itemize}


\underline{Theorem 1.4}

The following three conditions are equivalent:

\begin{enumerate}

\item Transitive Rationalizability
\item Condition $\alpha$ and $\beta$
\item WARP
\end{enumerate}

\textbf{\emph{Proof:}}


 %test


\end{document}