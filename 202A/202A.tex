\documentclass{article}

\usepackage{amsmath,amssymb,cancel}
\usepackage{hyperref}
\usepackage{qtree}
\newcommand{\RR}{\mathbb{R}}
\newcommand{\X}{\mathcal{X}}
\newcommand{\R}{\mathcal{R}}
\newcommand{\Lm}{\mathcal{L}}
\newcommand{\Om}{\mathcal{O}}
\newcommand{\core}{C_{f}(\rho)}



\title{SS 202A: }
\author{George Vega Yon\and Nicholas Adams-Cohen}

% Some useful macros
\def\reals{\mathbb{R}}
\def\noR{\cancel{R}}

\begin{document}
\maketitle


\section*{Lecture 1: Condorcet Paradox. Sept 30th, 2014}

\subsection*{Example 1: Social Choice Theory}
Ben, Matt, and Alex are on a search comittee to hire the next faculty member. They have the following preferences: 

\begin{table}[h]
\centering
\begin{tabular}{llll}
                                         & Alex                            & Matt                            & Ben                             \\ \cline{2-4} 
\multicolumn{1}{l|}{Political Scientist} & \multicolumn{1}{l|}{\textbf{1}} & \multicolumn{1}{l|}{\textbf{3}} & \multicolumn{1}{l|}{\textbf{2}} \\ \cline{2-4} 
\multicolumn{1}{l|}{Theorist}            & \multicolumn{1}{l|}{\textbf{2}} & \multicolumn{1}{l|}{\textbf{1}} & \multicolumn{1}{l|}{\textbf{3}} \\ \cline{2-4} 
\multicolumn{1}{l|}{Econometrician}      & \multicolumn{1}{l|}{\textbf{3}} & \multicolumn{1}{l|}{\textbf{2}} & \multicolumn{1}{l|}{\textbf{1}} \\ \cline{2-4} 
\end{tabular}
\end{table}

\textbf{Note:} Every actor is rational in that the have a clear rank order. But could a collective decision be reached?

Under \underline{Majority Rule}:
\begin{itemize}

\item (Alex \& Ben) P $>$ T
\item (Matt \& Alex) T $>$ M
\item (Ben \& Matt) M $>$ P
\end{itemize}

We define \underline{social ranking} as $xPy$ if and only if a majority prefer x to y. 
\emph{\textbf{But:}}

\begin{align*}
P &> T \\
T&>M\\
M&>P
\end{align*}

Which violates transitivity.

\subsection*{Example 2: Game Theory}

What if you create some institution with some rules and structure that will allow us to determine what people will do? 

\textbf{\underline{Binary Agenda Setting:}}

We have a sequence of votes of $x$ vs. $y$ which obeys the following rules:

\begin{itemize}

\item Finite.

\item The winner of round $t$ is one of the alternatives in round $t+1$.

\item The winner of the final round prevails.
\end{itemize}
\textbf{Tree:} If Alex was the agenda-setter, begin with Metrics vs. Political Science. If Metrics prevails, in the second round the vote will be Metrics vs. Theory, and Theory will win the vote, so a vote for Metrics in the first round is \emph{in actuality} a vote for Theory. If Political Science prevails, it will be a vote of Political Science vs. Theory, and Political Science will win, so a vote for Political Science in the first round is actually a vote for Political Science.

Thus, by setting the agenda in sequential voting, Metrics vs. Political Science is really a vote for Theory vs. Political Science and Political Science prevails.  %Insert Tree if time

\emph{*Takeaway:} By adding structure, we can form a point prediction. By manipulating what is voted for in the first round, you can determine the victor in the second round.

However, the agenda-setting power is limited. Consider the following preferences:

\begin{table}[h]
\centering
\begin{tabular}{llll}
                                         & Alex                            & Matt                            & Ben                             \\ \cline{2-4} 
\multicolumn{1}{l|}{Political Scientist} & \multicolumn{1}{l|}{\textbf{1}} & \multicolumn{1}{l|}{\textbf{3}} & \multicolumn{1}{l|}{\textbf{3}} \\ \cline{2-4} 
\multicolumn{1}{l|}{Theorist}            & \multicolumn{1}{l|}{\textbf{2}} & \multicolumn{1}{l|}{\textbf{1}} & \multicolumn{1}{l|}{\textbf{2}} \\ \cline{2-4} 
\multicolumn{1}{l|}{Econometrician}      & \multicolumn{1}{l|}{\textbf{3}} & \multicolumn{1}{l|}{\textbf{2}} & \multicolumn{1}{l|}{\textbf{1}} \\ \cline{2-4} 
\end{tabular}
\end{table}

Now, though Alex still prefers Political Science, there is nothing he can do as agenda setter to get this outcome. 

\subsection*{Example 3: How to choose an Agenda Setter?}

In the previous example, Alex was ``given'' as the agenda setter \dots but how was he choosen? 
Common approach is to make this a 2-stage process: 
\begin{enumerate}

\item How to choose an agenda-setter?
\item How will everyone vote? 
\end{enumerate}

We can write out everyone's preferences for an agenda-setter\dots but we quickly realize that preferences over an agenda-setter (who have the ability to determine the next faculty member) become peoples preferences over the next faculty member. That is:

\begin{table}[h]
\centering
\begin{tabular}{llll}
                                         & Alex                            & Matt                            & Ben                             \\ \cline{2-4} 
\multicolumn{1}{l|}{Alex A.S. = (P)} & \multicolumn{1}{l|}{\textbf{1}} & \multicolumn{1}{l|}{\textbf{3}} & \multicolumn{1}{l|}{\textbf{2}} \\ \cline{2-4} 
\multicolumn{1}{l|}{Matt A.S. = (T)}            & \multicolumn{1}{l|}{\textbf{2}} & \multicolumn{1}{l|}{\textbf{1}} & \multicolumn{1}{l|}{\textbf{3}} \\ \cline{2-4} 
\multicolumn{1}{l|}{Ben A.S. = (M)}      & \multicolumn{1}{l|}{\textbf{3}} & \multicolumn{1}{l|}{\textbf{2}} & \multicolumn{1}{l|}{\textbf{1}} \\ \cline{2-4} 
\end{tabular}
\end{table}

So choosing an institutional structure ``inherets'' the previous preferences. So we are going to need \emph{more} knowledge/ assumptions about the institutional structure.

\emph{*Takeaway:} Math alone does not give us the answer.

\subsection*{Three Themes of the Course}
\begin{enumerate}

\item \underline{Will of Society} is often a meaningless concept.
\item Adding \underline{institutions} can help\dots
\item But only to a point. \underline{Assumptions} matter, and you need them.
\end{enumerate}%test

\section*{Lecture 2}


When will underlying preferences lead to rationaliable choices?

$X \equiv \text{ set of alternatives } |X| \geq 3$. 

$\X \equiv$ set of all subsets $(X\setminus \emptyset)$ or $(2^\X \setminus \emptyset)$ (powerset). 

$S\subset X \Leftrightarrow \in \X$.

\bigskip

\underline{Derive choices from Weak Preference Relation}

M(R,S) = $\{x\in S:\forall y \in S, xRy\}$)\footnote{R is a preference, $S \in \X$.}

Minimal expected condition ``rationality'' is that the underlying preference M(R,S) $\neq \emptyset (\forall S \in \X)$

\subsection*{Three {roperties of the W.P. Relation}}


\begin{enumerate}

\item R transitive iff $\forall x,y,z \in X, xRy, yRz \Rightarrow xRz$ ``weak orders.''
\item R is quasi-transitive $\forall x,y,z \in X, xPy, yPz \Rightarrow xPz.$
\item Acylicity: $\forall x,y,z,\dots w,v$ then $xPyPz \dots wPv \Rightarrow xRv$
\end{enumerate}

\underline{Theorem 1.1}
Acylity is necessary and sufficient for maximal sets to be non empty $\forall \; s\in \X$

\textbf{\emph{Proof:}}

\underline{Necessity} $(\Leftarrow)$ (by contradiction.)

We know that the maximal set is not empty. 
Assume $xPyPz\dots uPv$ but $\neg(xRv)$.

So $\neg(vRu) \dots \neg(zRy),\neg(yRx)$ by the definition of P, and $\neg(xRv)$ by assumption.

But then M(R,$ \{ \cdot,\}$) is empty. Which is a contradiction.

\underline{Sufficiency}
\begin{enumerate}
\item If acylity is satisified, then pick $S \in \X$. 
\item Pick an arbitrary element $x_1 \in S$.
\item Compare $x_1$ with $S\setminus\{x_1\} $. If $x_1 R x_i (\forall i)$, we are done.
\item If $\neg (x_1 R y) \; \exists x_2 \in S$ such that $x_2Px_1$.
\item If $x_2 R x_i (\forall i)$, again we are done. 
\item If not you can follow the above steps and eventually find an $x' R x_i (\forall i)$. Hence, the set is acylic.  
\end{enumerate}


---


Choice function is a \underline{map} $C: \X \to \X$. That is for every sebset, $C$ gives you a subset of the subset. Ex $C(\{x,y,z\}) = \{x,y\} $.

\bigskip

Choice function satisfies \textbf{transitive rationalizability} iff $\exists$ a binary relation s.t. $S\in \X$, $C(S)=M(R,S)$ and moreover R is transitive.

\bigskip

\underline{Conditions:}


\begin{itemize}

\item Condition $\alpha$
\item Condition $\beta$
\item Weak Axiom of Revealed Preferences
\end{itemize}

-Does sensible things when you add and subtract alternatives.

\begin{itemize}

\item 
Condition $\alpha$: condition consistency property.

\[
\forall S,T; S\subset T \;\;\;C(T) \cap S \subset C(S)
\]

If I pick a bigger set and shrink it to a smaller set with choice still there, you will still pick it in a smaller set.

\item 
Condition $\beta$: Expansion-consitency property.

$S \subset T  \;\;\; C(S)\cap C(T) = \emptyset \Rightarrow C(S) \subset C(T) $


\item WARP

If $x\in C(S)$ and $y \not \in C(S)$ and $y \in C(T) \Rightarrow x \not \in T$ 

\end{itemize}


\underline{Theorem 1.4}

The following three conditions are equivalent:

\begin{enumerate}

\item Transitive Rationalizability
\item Condition $\alpha$ and $\beta$
\item WARP
\end{enumerate}

\textbf{\emph{Proof:}}

\section*{Lecture 5: October 14, 2014}
 %End of GS Proof
 
 \subsection*{Properties of Aggregation Rules}
 \begin{itemize}
 
 \item Decisive:

 \[
\rho\hat{\rho}|_{(x,y)} P(x,y; \rho) = P(x,y; \hat{\rho}), \text{ then } xPy \Rightarrow x\hat{P}y
 \]


 \item Monotonicity:

 \[
P(x,y;\rho)\subseteq P(x,y; \rho') \text{ \textbf{\underline{\&}} } R(x,y;\rho)\subseteq R(x,y; \rho') \text{ and } xPy ]  \text{ imply } xP'y
 \]


 \item Positive Responsiveness:

An aggregation rule is \textbf{positively responsive} if and only if, for all profiles $\rho,\rho' \in \R^n$ and all $x,y \in X$, if $xRy, P(x,y;\rho) \subseteq P(x,y;\rho')$ and $P(y,x; \rho)\supseteq P(y,x; \rho')$ with at least on inclusion being strict, then $xP'y$. 
\item Neutral if and only if, given $(\rho, \hat{\rho}, x,y,a,b) $:

\begin{align*}
P(x,y;\rho) &= P(a,b; \hat{\rho}),\\
\&R(x,y;\rho) &= R(a,b; \hat{\rho}),\\
xRy &\Rightarrow aRb
\end{align*}

\item Anonymity:

\[
G:N\to N|\rho \Rightarrow \rho' \text{ s.t. } R'_i = R_{G(i)} \text{ then } f(p) = f(p')
\]

 \end{itemize}

  \subsection*{Simple Rules}

  A decisive coalition is defined:

  \[
  L \subseteq N \Leftrightarrow \forall x,y;\; \; xP_iy \; \; [\forall i \in L ]\Rightarrow xPy
  \]

  $f \rightarrow L(f)$ is the set of decisive coalitions.

  Example: if dictorial, the decisive coalition is \underline{any} coalition with the dictator. That is (\{1\},\{1,2\},\{1,3\},\{1,2,3\}) = $L(f)$

This set $L(f)$ is:

\begin{itemize}

\item Monotonic:

\[
L \subseteq L', \text{ then } L \in L(f) \Rightarrow L'\in L(f)
\]


\item Properness:

\[
L \in L(f) \Rightarrow [N\setminus L ]\not \in L(f)
\]
\end{itemize}

\underline{Theorem:}
A rule $f$ is \emph{simple} if and only if it is \textbf{decisive}, \textbf{neutral}, and \textbf{monotonic}.

\underline{Proof:}
WTS: $f = f_{L(f)}$. We show this in two parts:

\begin{enumerate}

\item $xPy \Rightarrow xP_{L(f)}y$,
\item $ xP_{L(f)}y \Rightarrow xPy$.
\end{enumerate}
\bigskip
\begin{enumerate}

\item WTS:  $xPy \Rightarrow xP_{L(f)}y$.

That is, we want to demonstrate that if $xPy$, it is formed by a decisive coalition $L \in L(f)$. This proof proceeds by starting with some $\rho$, transforming the profile in several steps in order to find $\hat{\rho} $. That is, transform $\rho \to \rho^1 \to \rho^2 \dots \to \hat{\rho} $.

We will show this by an example with seven individuals.

\begin{itemize}

\item $\rho$: A profile such that $xPy$, with a group G such that $\forall i \in G, xP_iy$. Individuals in $j \in [N\setminus G]$ either $xI_jy$ or $yP_jx$.

\item $\rho'$: Define $rho'$ such that 
\begin{align*}
P(x,y;\rho) &= P(w,v; \rho'),\\
\&R(x,y;\rho) &= R(w,v; \rho'),\\
\end{align*}

Because simple rules are neutral, it must be that $wP'v$. Further, note that we still have $\forall i \in G, wP_iv$.

\item $\rho''$: Change the profile such that some individuals  $j \in [N\setminus G]$ for whom $wI_jv$ are now $wP_jv$. By monotonicity, $wP''v$.


\item $\rho'''$: Any individual $j$ such that $\neg(wP_jv)$ may change their preference arbitrarily under $\rho'''$. Because the aggregation rule is decisive, and $P(w,v; \rho'') \subseteq P(w,v; \rho''')$ and $wP''v, wP'''v$.

Further, it is still true that for group G, $\forall i \in G, wP_iv$. 

\end{itemize}

Thus, we show that G is a decisive coalition: regardless of the transformation, group G remains a decisive coalition and their preference represents the preference choosen under the aggregation rule. Thus, $f = f_{L(f)} $.



\item WTS:  $ xP_{L(f)}y \Rightarrow xPy$.

$\exists L \in L_{(f)}$ such that $xP_iy [\forall i \in L]$, so $xP_{L(f)}y$. Then, by the definition of decisive coalition, $\Rightarrow xPy$.

\end{enumerate}


\section*{Lecture 6: October 16, 2014}

\subsection*{Voting Rules}

 A  ``pair'' coalition is group $(S,W)|_{(x,y)}$ where $i \in S: xP_iy$ and $j \in W: xR_iy$.e say $S$

\underline{Definition:} \emph{Decisive Pair:}  $D(f) = (S,W)$ where $S\subseteq W$ such that:

\[
\{xP_iy [\forall i \in S] xR_iy [\forall i \in W]\} \Rightarrow xPy
\]

\underline{Definition}: \emph{Voting Rule}: $f \equiv f_{D(f)}$


\underline{Theorem:}
A rule $f$ is a \emph{voting rule} if and only if it is \textbf{neutral}, and \textbf{monotonic}.


\subsection*{Counting Rules}

Definition: The subset of voting rules that are \underline{anonymous}.

\begin{itemize}

\item In voting rules, we defined $(S,W) \in 2^N \times 2^N$.
 
\item Now, define $(p,a) \in NxN$, where $p=$ pros $a=$ antis.

\item  A pair of numbers is a decisive pair of numbers in $f$, denoted $N(f)$ if and only if $(x,y,\rho) $:

\begin{align*}
|P(x,y;\rho) &\geq pros|\\ %%Check this with book
|P(y,x;\rho) &\leq antis|\\
\end{align*}

\item N is \textbf{monotonic} $\Leftrightarrow p,a \in N \Rightarrow (p',a') \text{ where } p'\geq p, a' \leq a \in N$.
\item N is \textbf{proper} if $(p,a)$ such that $a\geq p$ then $(p,a) \not \in N$ 

% \underline{Proof}: 

% Observe the following preference profile with 5 individuals:

% \begin{table}[h]
% \begin{tabular}{lllll}
% x & x & y & y & y \\
% y & y & x & x & x
% \end{tabular}
% \end{table}

% Now claim $(p,a) = (2,3)$ is a decisive pair where $a>p$, such that xPy.


\end{itemize}

The \textbf{definition of a counting rule} is:

\[
f=f_{N{(f)}}
\]

Note: 
\begin{itemize}

\item [--] Simple rules are \textbf{anonymous} if and only if they are \textbf{q-rules}.

\item [--]Counting rules are \textbf{decisive} if and only if they are \textbf{q-rules}.
\item [--] \textbf{May's Theorem:} A rule is neutral, anonymous, and positvely responsive if and only if it is a plurality rule.
\end{itemize}


\section*{Lecture 7: October 21, 2014}

\textbf{Definition:} The \emph{collegium} of a set of decisive calitions $\Lm(f)$ is:

\[
K(\Lm(f)) \equiv \cap_{L\in\Lm(f)}L
\]

\begin{itemize}

\item For a \underline{simple rule} if $\cap L_f \neq \emptyset$\dots we say the rule is ``collegial''. 

$\Rightarrow$ If you are in the collegium, society cannot have a strict preference \underline{unless} $i \in L_f$ also has the preference.

%Check sticky

\item A dictorial simple rule has a collegium of 1 \dots the dictator.
\end{itemize}

Observe: if $f$ is collegial, then $f(\rho)$ is acylic $\forall \rho \in \R^n$.

\begin{itemize}

\item To have acyclicity requires $x_1Px_2P\dots Px_{k-1}Px_k$ and $x_kPx_1$
\item To get this in collegial simple rule would require $i \in K$ to have acyclic preferences.
\item Collegial rules are   ``well behaved'' in the sense that acyclic $\Rightarrow$ non-empty maximal sets.
\item $\star$ \textbf{BUT:} Collegial rules are also undesirable in the sense that you have a group fo dictators.
\end{itemize}

\subsection*{Nakamura Numbers and Acyclicity}

\textbf{Definition:} The \emph{Nakamura Number} for a simple rule, s(f) is:

\begin{itemize}

\item $\infty$ if $f$ is collegial.
\item Cardinality of the smallest non-collegial sub-family of decisive coalitions.
\end{itemize}

Example: Majority Rule:

\[
[\{1,2\}, \{1,3\}, \{2,3\}, \{1,2,3\}]
\]
\[
[\{1,2\}, \{1,3\}, \{2,3\}, \xcancel{\{1,2,3\}}] 
\]

Removing $\{1,2,3\}$ does not make it collegial, so for majority rule the $s(f)$ is exactly three.\footnote{Unless n=4, in which case $s(f)=4$.}

\textbf{Note}: $s(f) \in \{3,\dots,n\}$.

\bigskip

\textbf{Theorem:} A simple rule is acyclic if and only if $|X| < s(f)$.

Proof: (Sufficency)
$|X| < s(f) \Rightarrow$ Acyclic.

\begin{itemize}

\item Suppose we have $|X| < s(f)$ but a cycle in societal preferences.

That is, we have:

\[
x_1 P x_2 \dots Px_{k-1}Px_kPx_1
\] 

We know $k \leq |X|$ and so $k<s(f)$.


\item Because $f$ is a simple rule, we must have decisive groups such that:

\[
\{ P(x_1,x_2; \rho), P(x_2,x_3; \rho) \dots P(x_k,x_1;\rho) \} \in \Lm(f)
\]


\item Given that we know $k < s(f)$, we know there are less decisive coalitions than what is minimially possible to allow for a non-collegial set of decisive coalitions\dots and so there must be one individual in each of the decisive coaltions. 
\item But then this individual has cyclic preferences\dots a \textbf{contradiciton}.
\end{itemize}

\begin{itemize}

\item If f is a q-rule, $s(f) = \left[\frac n {n-q}\right]$.

\item Becuase the Nakamura Number of majority rule is ``roughly'' three (unless n=4), then you run into problems as soon as there are $\geq 3$ alternatives. You run into condorcet cycles.
\end{itemize}

\section*{ Oct 23: Restricting Preference Profiles}

\subsection*{Single Peaked Preference}

An individual $i$ is going to have \emph{single-peaked} preferences if we do the following:

\begin{itemize}

\item 
Create some order $Q$ of alternatives, where $yQx$ is interpreted as ``$x$ is to the left of y'' on a number line. 
\item Label alternatives $a_1,a_2 \dots a_r$ where $r = |X|$ such that $a_rQa_{r-1}\dots Qa_2Qa_1$
\item We say that $i's$ preferences are single peaked with respect to an ordering $Q$ if and only if $\exists$ t such that:
  \begin{itemize}
  
  \item $a_t P_i a_{t+1} \dots a_{r-1}P_ia_r$
  \item $a_t P_i a_{t-1} \dots a_{2}P_ia_1$
  \end{itemize} 
\item We say $a_t$ is $i's$ ``ideal point'' or \underline{strictly} most preferred alternative
\item A profile $\rho$ has single peaked preferences if there exists a singe $Q$ ordering such that $R_i$ is single peaked with respect to $Q$ for all $i$.
\end{itemize}

\bigskip

\textbf{Theorem:} For all voting rules $f$, $f(\rho)$ is Quasi-Transitive for all $\rho \in S$.

\bigskip

\textbf{Define:} The \emph{``core''} of a preference aggregation rule $f$ as 

\[
M(f(\rho), X) \equiv C_f(\rho)
\]

\underline{Characterizing the Core}
% \begin{itemize}

% %STICKY
% % \item Fpr a simple rule $f$, an alternative in the chore is characterized:

% % \[
% % y \in C_f(\rho) \Leftrightarrow z \neq y, L \in \Lm(f) \text{ s.t. } zPiy [\forall i \in \Lm]
% % \]
% \end{itemize}

\underline{F-medians}

\begin{itemize}

\item Denote $x_i$ i's strictly best alternative
\item 
\begin{align*}
z\in X: \;\;\; &L^{-}(z)=\{i:zQx_i\} \\
&L^{+}(z)= \{i:x_iQz\} 
\end{align*}

\item The ``f-median'' is an alternative $z$ such that 
\begin{align*}
L^{-}(z) &\not \in \Lm (f)\\
L^{+}(z)&\not \in \Lm (f)\\
\end{align*}

\subsection*{Order Restriction}

Can you reorder the \textbf{individuals} such that for \emph{any} pair of alternatives $(x,y)$ satisfy the following property:

\[
P(x,y;\rho), I(x,y;\rho), P(y,x;\rho)
\]

Can put along a \textbf{single} line. Formally, if there exists an ordering of individuals $G(i): N \to N$, a set of preferences satisfies order restriction if and only if $\exists G(i)$ such that either:

\[
\{G(i)\in N:xP_iy\} \geq \{G(i)\in N:xPI_iy\} \geq \{G(i)\in N:yP_ix\}
\]

OR

\[
\{G(i)\in N:xP_iy\} \leq \{G(i)\in N:xPI_iy\} \leq \{G(i)\in N:yP_ix\}
\]

\textbf{Thereom:} Let $f$ be a voiting rule; then for all $\rho \in \Om, f(\rho)$ is quasi-transitive.


\end{itemize}

\section*{Lecture 9: October 28}
\underline{The Core}
\begin{itemize}
  \item $C_{f(\rho)} = M(f(\rho),X)$

  \item If $X<\infty$, for a simple rule:
  \begin{itemize}
    \item $f(\rho)$ is acylic when $\rho \in \mathcal{S}$.
    \item $\core \neq \emptyset $ is characterized by f=medians
    \item $\core$ of a voting rule $\Leftrightarrow$ $C_{f_{L(f)}}(\rho)$
  \end{itemize}
  \item Further, we know: 
  \begin{itemize}
  
  \item When $f(\rho)$ is acylic, you have non-empty maximal sets $\Rightarrow \core \neq \emptyset$ 
  \item For simple rules, $\core \neq \emptyset$ as long as $|X| < s(f)$
  \item For order restrictions ($\rho \in \Om$), then $f(\rho)$ for a voting rule is Quasi-Transitive $\Rightarrow \core \neq \emptyset$ 
  \end{itemize}
\end{itemize}



\begin{itemize}

\item If $f$ is simple, 

\[
y \in \core \Leftrightarrow \not \exists x\neq y \;\;\&\;\; L\in\Lm(f): xP_iy [\forall i \in L]
\]

That is, the core is the stuff that is `unblocked.'' It \textbf{is} blocked if all members of a decisive group prefer some alternative.
\end{itemize}

\subsection*{Lower and Upper Contour Sets}

\begin{align*}
P_i^{-1}(x) &= \{y\in X: xPy\}\\
P_i^{1}(x) &= \{y\in X: yPx\}
\end{align*}

$P_i^{-1}(x)$ is the lower (strictly worse than) set.
$P_i^{1}(x)$ is the upper (strictly better than) set.

\begin{align*}
P_L^{-1}(x) = \cap_{i\in L} P_i^{-1}(x); \;\;\; & \;\;\;  P_L^{1}(x) = \cap_{i \in L} P_i^{1}(x)\\
P_f^{-1}(x) = \cup_{L\in \Lm(f)} P_L^{-1}(x); \;\;\;&\;\;\;  P_f^{1}(x) = \cup_{L\in \Lm(f)} P_L^{1}(x)\\
\end{align*}

\textbf{NOTE:}

\[
x \in \core \Leftrightarrow P_f(x) = \emptyset
\]

Definition: the Pareto Set

\[
PS_L(\rho) = \{x\in X: \forall y \in X; \;\;\;\;yP_ix [i \in L] \Rightarrow xP_iy [i \in L] \}
\]

Alternatively, y is \textbf{not} in the Pareto set if:

\[
y \not \in PS_L(\rho) \Rightarrow \exists x \in X: xR_iy [\forall i \in L], xP_jy[ j \in L]
\]

\begin{itemize}

\item Claim: If $x \in PS_L(\rho)\Rightarrow P_L(x) = \emptyset$

\textbf{Proof:} Assume $x \in PS_L(\rho)$ but $P_L(x) \neq \emptyset$. Then $\exists y$ such that $yP_ix \forall i\in L
.$ However, if this is true $x \not\in PS_L(\rho)$


\item \emph{Note:} the convervse is not true. For example, if $\{x,y\} = X, N=2$ and the two profiles are $yP_ix$ and $yI_jx$, then the pareto set is $y$ but the upper contour set is $\emptyset$.

\item We thus can say for a coalition, the pareto set is a subset of things with empty upper contour sets:

\[
PS_L(\rho) \subseteq \{x\in X: P_L(x) = \emptyset\}
\]

\end{itemize}

\begin{itemize}

\item Claim: $\core \equiv \cap_{L\in L(f)} PS_L(\rho) $


\item Show: if $f$ is simple, then 
\[
\cap_{L\in \Lm(f)} PS_L \subseteq C_f(\rho)
\]

\textbf{Proof:}
 $x \in PS_L(\rho)$ of an $L\in \Lm(f) \Rightarrow P_L(x) = \emptyset$. Ergo, $P_f(x) =  \cup_{L\in L(f)} P_L(x) = \emptyset \dots x\in\core$.
 \item Now, if $|X| < \infty; \rho \in S ; f$ simple, then:

\[
C_f(\rho) \subseteq \cap_{L\in \Lm(f)} PS_L 
\]

\textbf{Proof:}

Prove the contrapositive: suppose $x \not \in \cap_{L\in \Lm(f)} PS_L$.

Then, there exists an $L$ and $y \neq x$ such that 

\begin{align*}
yR_ix \;\;\;\;\;\;&\forall i\in L\\
yP_jx \;\;\;\;\;\;&\text{ for some } j\in L
\end{align*}

If on the realm of single peaked preferences, this means there exists $z$ such that $zP_i x [\forall i \in L] \Rightarrow zPX \Rightarrow x \not \in \core$
\end{itemize}
\section*{Lecture 10: October 30}
\subsection*{Spatial Model}

\textbf{\underline{Outline:}}
\begin{itemize}
  \item Consider Single Peaked Prefereence $\rho \in S$ in $\RR^1 \equiv [0,1]$

  This is a strict quasi-concave utility:

  \[
  u(\lambda x + (1-\lambda) y > \min\{u(x),u(y)\})
  \]

\item There is an analog to Single-Peakedness in $\RR^k$. \textbf{However}, in general S.P. in $\RR^N$ is \underline{not} sufficient to ensure non-empty cores.
\item We want to say something about dimensionality of space and core existence.
\item There are results that represent the Nakamura number and the $k$ that supports non-empty cores.
\item Consider k's and rules such that core non existence is \textbf{not} assured, and then consider \textbf{new} conditions for core existence.

\item \textbf{In the end}: we find for Majority Rule, there is a ``genericity'' of core nonexistence when $k>2$.

\underline{We make the following assumptions:
}
\item $X \equiv $ [closed and bounded]
\item $R_i$ are continuous weak orders. Continuity is achieved if $P_i^{-1}(x) and P_i(x)$ are open for all $i \in X$ for all $x$.

\end{itemize}


\subsection*{Spatial Model: Analog to Single-Peaked Preferences in $\RR^k$}

\underline{Strict Convexity}

\begin{itemize}

\item Preference relation $\R$ is strictly convex if and only if $\forall x \neq y$ and $xR_iy$, 

\[
[\lambda x + (1-\lambda)y] P_i y \;\;\; [\forall \lambda \in [0,1]]
\]
\item Then $\rho \in \RR^n_{cs}$ such that all $R_i$ are strictly convex.


\item Note: if preferences are continuous, then for a utility function $u:(x)$ is strictly quasi concave:

\[
u(\lambda \cdot x + (1-\lambda) y) > \min \{u(x),u(y)\})
\]

\item This yields \textbf{strictly convex} upper contour sets

\textbf{NOTE:} 

Stricty Convexity $\Leftrightarrow$ Single peakedness in $\RR^k$ $\Leftrightarrow$ convexity of $P_i(x)$

\end{itemize}

\bigskip

\begin{itemize}

\item There is an analogy between 
\begin{itemize}

\item $\rho \in \R^n_{CS} $ with $X\subset \RR^K$ and,
\item  $\rho \in S^n$ with $|X|<\infty$
\end{itemize}
\end{itemize}

\underline{Euclidien Preferences}: $=(||x_i-x|| )^2$.
\begin{itemize}

\item Pareto set $PS_L(\rho)$ of a coalition L is the Convex Hull of $Y$ (Con$Y$) where $Y\equiv \{x_1,x_2 \dots, x_k\}$.

\item The unique core point is the intersection of the convex hull/pareto set.

\subsection*{Core (Non) Empty}
In the finite world, $|X|<\infty$, not restrained to S.P.P, then the core is non empty if and only if $|X|<s(f)$. Thus, for majority rule where $s(f)$ generall equals 3, you can have at most two alternatives to ensure a non-empty core.

For the continuos world, where $k\equiv$ dimensionality:

\begin{itemize}
  \item $k=1, \rho \in \R_{CS}$, then regardless of $s(f)$ core non-empty. 
  \item If $k>1$, even with $\rho \in \R^n_{CS}$, we might have an empty core.
\end{itemize}


\end{itemize}




\end{document}