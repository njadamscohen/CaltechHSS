\documentclass{article}

\usepackage{amsmath,amssymb,cancel}
\usepackage{hyperref}

\title{SS 201A: Analytical Foundations of Social Science}
\author{George Vega Yon\and Nicholas Adams-Cohen}

% Some useful macros
\def\reals{\mathbb{R}}
\def\noR{\cancel{R}}

\begin{document}

\maketitle

\section*{Sep 29th, 2014}

\subsection*{Why study choice theory?}

The agent has a utility function and maximizes $\max_{x\in X}U(x)$, where $U$ is
a utility function and $X$ is the set of available actions.

\emph{What is the theoretical foundation of this model?} is precisely what this
class is about.

Our basic tool is a preference relation $\prec$ or $R$. ``$x\prec y$'': agent 
prefers $y$ to $x$. More formally, the result we are going to obtain is: 
``$\prec$'' satisfies  axioms 1,...,$n$ iff there exists a function 
$U:X\to\reals$ s.t. $U(x)<U(y) \iff x\prec y$.

\begin{align*}
l&=(x_1;p;x_2;1-p) \\
r&=(y_1;q;y_2;1-q)
\end{align*}

$l\prec r \iff pU(x_1) + (1-p)U(x_2) < qU(y_1) + (1-q)U(y_2)$: von 
Neumann-Morgenstenn Expected utility theory.

\subsection*{Representation of Weak Order}

Assuming $X$ is a countable set:

\begin{itemize}
\item $\prec$ is \underline{weak order} iff $\exists U:X\to\reals$ s.t. $x\prec y
\iff U(x)<U(y)$.

\item \underline{Binary relation} on a set $Y$ is a set of ordered pairs $(x,y)$
with $x,y\in Y$.

\item \underline{Notation}:

\begin{align*}
R&:\text{Binary relation} \\
\prec&:\text{Preference relation}\\\\
xRy&: \text{ means }(x,y) \in R \\
x\noR y&:\text{ means }(x,y)\notin R
\end{align*}

\item \underline{Remark} if $R$ is a binary relation on $Y$, exactly one of the
following is true:
\begin{enumerate}
\item $xRy \wedge yRx$ or 
\item $xRy \wedge y\noR x$ or 
\item $x\noR y \wedge yRx$ or 
\item $x\noR y \wedge y\noR x$

\end{enumerate}
\end{itemize}

\noindent {\bf Example} $Y$ is the set of all living people, and $R_1:$``is shorter
than'' ($<$).

\noindent {\bf Definition} A binary relation $R$ on $Y$ is:
\begin{enumerate}
\item Reflexive if $xRx \forall x\in Y$.
\item Irreflexive if $x\cancel{R}x \forall x\in Y$.
\item Symmetric if $[xRy \vee yRx] \forall x,y\in Y$.
\item Asymmetric if $[xRy\implies x\noR y \forall x,y\in Y]$.
\item Antisymmetric if $[(xRy \wedge yRx)\implies x=y] \forall x,y\in Y$.
\item Negatively transitive if $[(x\noR y \wedge y\noR z)\implies x\noR z] \forall
x,y,z\in Y$.
\item Connected (complete) if $[xRy \vee yRx] \forall x,y\in Y$.
\item Weakly connected if $x\not=y\implies xRy \vee yRx\forall x,y\in Y$.
\end{enumerate}

\noindent {\bf Exercices} Show the following statements:
\begin{enumerate}
\item Asymmetric binary relation is irreflexive.

\noindent {\bf Proof} Assume a relation which is not irreflexive, then $(x,x)\in
R \implies (xRx\implies x\noR x)$, which is a contradiction.

\item An irreflexive and transitive binary relation is asymmetric.

\noindent {\bf Proof} If $R$ were not asymmetric it would be true that $xRy\implies yRx$, then, by
transitivity, $xRx$, this is, reflexivity, which is a contradiction.

\item A binary relation $R$ is neg. trans. iff $\forall x,y,z\in Y (xRy)\implies
(xRz \vee zRy)$.

\noindent {\bf Proof} By the conterpositive of $x\noR y \wedge y\noR z\implies x\noR z$, we know that
$xRz\implies xRy \vee yRz$. At the same time the counterpositive of $xRy\implies
xRz \vee zRy$ implies that $x\noR z \wedge z\noR y \implies x\noR y$.

\item Find all the properties of $R_1$ (shorter than).

\noindent {\bf Sol}
\begin{enumerate}
\item It can't be reflexive, ($A$ can't be taller than itself), then
\item It is irreflexive.
\item $R_1$ is a strict preference relation, thus it can't be symmetric, then
\item It is asymmetric.
\item No
\item It is negatively transitive.
\item Irreflexivity implies that is not connected, furthermore,
\item It is weakly connected.
\end{enumerate}
\end{enumerate}
%test

\section*{Oct 1st, 2014}

\subsection*{Important binary relations}
\begin{itemize}
\item Weak order
\item Strict order iff weak order + weakly connected
\item Equivalence iff reflexivity + symmetry + transitivity
\end{itemize}

\noindent {\bf Equivalence class} $R(x):=\{y\in Y: yRx \forall x\in Y\}$.
Two equivalence classes are either identical or disjoint, i.e. $R(x)=R(y)
\iff xRy$

\noindent {\bf Proof}

\begin{itemize}
\item[\bf ($\Longleftarrow$)] Given $xRy$
\item[] $\forall z\in R(x), zRx$
\item[(transitivity)] $xRy\implies zRy\implies z\in R(y)$
\item[] $\implies R(x)\subseteq R(y)$
\item[] 
\item[] $\forall z\in R(y), zRy$
\item[transitivity + symmetry] $xRy\implies zRx\implies z\in R(x)$
\item[] $\implies R(y)\subseteq R(x)$
\item[] Then $R(y)=R(x)$
\item[] 
\item[\bf ($\implies$)] Given $R(x)=R(y)$
\item[] $z\in R(x) \wedge z\in R(y)$
\item[] $xRz \wedge yRz$
\item[Symmetry] $xRz \wedge zRx$
\item[Transitivity] $xRy$, now, given the previous proof
\item[] $R(x)=R(y) \iff R(x)\subseteq R(y) \wedge R(y)\subseteq R(x)$
\end{itemize}

\noindent {\bf Equivalence class of $Y$ under $R$} $\{R(x)\}_{x\in Y}$ will be
denoted as $Y/R$

\noindent {\bf Show} $Y=\bigcup_{A\in Y/R}A$

\noindent {\bf Proof}
\begin{itemize}
\item[\bf ($\supseteq$)] Let $A\in Y/R$
\item[] $\implies \exists a\in Y$ s.t. $A=\{x\in Y: xRa\}$
\item[] $\implies A\subseteq Y$ by def
\item[] $\implies \bigcup_{x\in Y/R}A\subseteq Y$
\item[]
\item[\bf ($\subseteq$)] Let $z\in Y$, then
\item[] $zRz$ by reflexivity
\item[] $\implies z\in R(z) \in Y/R$
\item[] $\implies z\in\bigcup_{A\in Y/R}A$
\item[] $\implies \bigcup_{A\in Y/R}A \supseteq Y$
\item[] Therefore $\implies Y=\bigcup_{A\in Y/R}A$
\end{itemize}

\subsection*{Preference Relation}

\noindent {\bf Indifference} $x\sim y\iff x\not\prec y \wedge y\not\prec x$.This
type of relation can be given by indifference, uncertainty or incomparability.

\noindent {\bf Weak preference} $x\precsim y \iff x\prec y \vee x\sim y$

\subsection*{Show}
\begin{enumerate}
\item $x\prec y \veebar x\sim y \veebar y\prec x$
\item $\prec$ is transitive
\item $\sim$ is an equivalence
\item ($x\prec y, y\sim z)\implies x\prec z \wedge (x\sim y, y\prec z)\implies
x\prec z$
\item $\precsim$ is transitive and connected.
\end{enumerate}

\noindent {\bf Proof (George)}
\begin{enumerate}
\item $x\prec y \veebar x\sim y \veebar y\prec x$
\begin{enumerate}
\item[\bf (case 1)] Assume $x\prec y\wedge y\prec x$, then
\item[Transitivity] $x\prec x$, which is a contradiction as $\prec$ is not symmetric
\item[\bf (case 2)] Assume $x\sim y \wedge y\prec x$, then
\item[By def] $x\sim y\implies x\not\prec y \wedge y\not\prec x$, then
\item[] as $y\prec x$, it would be a contradiction.
\item[\bf (case 3)] Assume $x\prec y \wedge x\sim y$
\item[By def] $x\sim y\implies x\not\prec y \wedge y\not\prec x$, then
\item[] as $x\prec y$, it is a contradiction. QED
\end{enumerate}

\item $\prec$ is transitive
\begin{enumerate}
\item[] If it is not transitive, then $\exists x,y,z$ s.t. $x\prec y\wedge y\prec
z \implies x\not\prec z$
\item[By def] $x\prec y\iff x\precsim y \wedge y\not\precsim x$
\item[] $y\prec z\iff y\precsim z \wedge z\not\precsim y$
\item[Transitivity] $x\precsim z$, which $\iff x\prec z \vee x\sim y$
\item[] But as $x\prec z$, then $x\sim y$ would be a contradiction,
\item[] Thereby, it should be true that $\prec$ is transitive. QED
\end{enumerate}

\item $\sim$ is an equivalence
\begin{enumerate}
\item[] Take any $x,y,z\in X$ s.t. $x\sim y \wedge y\sim z$, then
\item[Transitivity] $x\sim z$, and symmetry of $\sim$ implies $z\sim x$
\item[Transitivity] $x\sim x$, thus, as it must be true that $x=x$
\item[] $\sim$ should be an equivalence.
\end{enumerate}

\item (i) $(x\prec y, y\sim z)\implies x\prec z$ (ii) $(x\sim y, y\prec z)\implies
x\prec z$
\begin{enumerate}
\item[\bf (i)] $(x\prec y, y\sim z)\implies x\prec z$
\item[By def] $((x\precsim y \wedge y\not\precsim x)\wedge (y\precsim z \wedge 
z\precsim y))$
\item[Transitivity] $x\precsim z$, which 
\item[By def] $x\prec z \vee x\sim z$, but
\item[Symmetry \& Transitivity] $x\sim z\implies x\sim y$, it should be
\item[] true that $x\prec z$ QED.
\item[]
\item[\bf (ii)] $(x\sim y, y\prec z)\implies x\prec z$
\item[By def] $(x\precsim y \wedge y\precsim x) \wedge (y\precsim z\wedge
z\not\precsim y)$
\item[Transitivity] $x\precsim z$, which
\item[By def] $x\sim z \vee x\prec z$, but as by
\item[Transitivity] $x\sim z\implies y\sim z$, them it should be true that
\item[] $x\prec z$ QED.
\end{enumerate}

\item $\precsim$ is transitive and connected.

\begin{enumerate}
\item[\bf (transitive)]  $\exists x,y,z\in X$ s.t. $x\precsim y, y\precsim z$, then
\item[by def] $(x\prec y \vee x\sim y),(y\prec z \vee y\sim z)$
\item[] $(x\prec y, y\prec z) \vee (x\prec y,y\sim x)\vee(x\sim y, y\prec z) \vee (x\sim y, y\sim z)$
\item[] From the first and the last one we know that
\item[Transitivity] $(x\prec z \vee x\sim z)$, which, by def
\item[] $x\precsim z$
\end{enumerate}
\end{enumerate}

\end{document}
