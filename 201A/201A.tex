\documentclass{article}

\usepackage[colorlinks=true,linkcolor=blue]{hyperref}
\usepackage{amsmath,amssymb,amsthm}
\usepackage{cancel}

% THEOREM DEFINITIONS %
\theoremstyle{definition}
\newtheorem{mydef}{Definition}
\newtheorem{mythr}[mydef]{Theorem}
\newtheorem{mycor}[mydef]{Corollary}
\newtheorem{mypro}[mydef]{Proposition}
\newtheorem{myoth}[mydef]{}

% Useful macros %
\def\sand{\;\&\;}
\def\bspace{\mathcal{B}}
\def\tor{\text{ or }}
\def\tand{\text{ and }}
\def\tnot{\text{ not }}
\def\tiff{\text{ if and only if }}
\def\tforall{\text{ for all }}

% Format
\usepackage[margin=1in]{geometry}
\usepackage{setspace}
\onehalfspacing
\begin{document}
\section{Binary relations}
\noindent {\bf Definition } A binary relation $R$ on $Y$ is
\begin{itemize}
\item {\bf Reflexive} if $xRx$ for every $x\in Y$,
\item {\bf Irreflexive} if not $xRx$ for every $x\in Y$,
\item {\bf Symmetric} if $xRy\implies yRx$ for every $x,y\in Y$,
\item {\bf Asymmetric} if $xRy\implies\text{ not }yRx$ for every $x,y \in Y$,
\item {\bf Antisymmetric} if $(xRy,yRx)\implies x=y$ for every $x,y\in Y$,
\item {\bf Transitive} if $(xRy,yRz)\implies xRz$ for every $x,y,z\in Y$,
\item {\bf Negatively transitive} if $(\text{not }xRy,\text{ not }yRz)\implies\text{ not }xRz$, for every $x,y,z\in Y$,
\item {\bf Connected} if $xRy$ or $yRx$ for every $x,y\in Y$,
\item {\bf Weakly connected} if $x\not=y\implies (xRy\text{ or }yRx)$ for all $x,y\in Y$.
\end{itemize}

\noindent{\bf Remark:}

\begin{itemize}
\item An asymmetric binary relation is irreflexive.

\noindent{\bf Proof} 

Proof by contradiction, for any $x\in Y$ we define $xRx\implies$ not $xRx$, which is a contradiction. $\Box$

\item An irriflexive and transitive binary relation is asymmetric.

\noindent {\bf Proof}

Proof by contradiction, assume is not asymmetric, then for any $x\in Y$ $xRy$ and $yRx$,
now, take any $z$ s.t. $zRx$ and $xRz$, then, by transitivity we can get $zRy$ and $yRz$,
which leads us to $zRz$, but as $R$ is irreflexive then we have a contradiction.$\Box$

\item A binary relation  $R$ is negatively transitive iff for all $x,y,z\in Y$ $xRy\implies (xRz\text{ or }zRy)$

{\bf Proof}
\begin{align*}
\tnot xRy\tand \tnot yRz &\implies \tnot xRz \\
\text{Taking its contrapositive}\\
xRz&\implies xRy\tor yRz,
\end{align*}

which is what we where looking for. $\Box$
\end{itemize}

\section{Weak order}

\noindent {\bf Definition} A binary relation $R$ on a set $Y$ is
\begin{itemize}
\item A {\bf Weak order} iff $R$ is asymmetric and negatively transitive,
\item A {\bf Strict order} iff $R$ is weakly connected weak order,
\item An \textbf{Equivalence} iff $R$ is reflexive, symmetric and transitive.
\item A {\bf Strict partial order} iff $R$ is reflexive and transitive (from section \ref{sec:strictpartialorder}).
\end{itemize}

\noindent {\bf Definition} Let $R$ be an equivalence class on $Y$. The set

\begin{equation*}
R(x)=\sim(x)=\{y\in Y| \forall x\in Y, yRx \},
\end{equation*}

is called the equivalence class generated by $x$

\noindent{\bf Remark:}
\begin{itemize}
\item $R(x)=R(y)$ iff $xRy$

{\bf Proof}

{\bf ($\Rightarrow$)} $R(x)=R(y)$ implies $xRy$

By direct proof, by definition $\forall x' \in R(x)$ and $\forall y'\in R(y)$ implies $x'Rx$ and $y'Ry$ respectively, furthermore, as both sets are equal, then $x'Ry$. Now, as $R$ is symmetric we also know that $xRx'$ Therefore, by transitivity of $R$ we get $xRy$. $\Box$

{\bf ($\Leftarrow$)} $xRy$ implies $R(x)=R(y)$

By direct proof, let $R(x), R(y) \subset Y$ be the respective equivalence classes of $x$ and $y$. For any member $z$ of $R(x)$, by definition we know that $xRz$ and by symmetry of $R$ that $zRx$. Now, $R$ transitive implies that $zRy$, and again, by symmetry of $R$, $yRz$ which is a member of $R(y)$, thus, $R(x)\subseteq R(y)$. Repeat the same steps for any member of $R(y)$ and we will get $R(y)\subseteq R(x)$. Therefore, $R(x)=R(y)$. $\Box$

\item Any equivalence classes are either identical or disjoin

{\bf Proof}

Proof by contradiction. Assume not, then for any two $x,y\in Y$, let them have their equivalence classes $R(x)$ and $R(y)$ s.t. $R(x)\cap R(y)\not=\emptyset$ and $R(x)\not=R(y)$. Then, there should be some $z\in R(x)\cap R(y)$ s.t. $xRz$ and $yRz$, furthermore, by asymmetry and transitivity of $R$, $\forall w\in R(x), wRz$ and $zRw$ as well. This way, $R$ transitive implies that $R(x)\subseteq R(y)$, and what's more, by applying the same logic with any element of $R(y)$ we get $R(y)\subseteq R(x)$. Therefore, $R(x)=R(y)$, which is a contradiction. $\Box$

\item[] Notation: We denote $\{R(x)\}_{x\in Y}$ by $Y/R$
\item $Y/R$ is a partition of $Y$, that is (i) for any $A,A'\in Y/R$, if $A\not=A'$ then $A\cap A'=\emptyset$, (ii) $Y=\cup_{A\in Y/R}A$.

{\bf Proof}

{\bf (i)} Already proof. $\Box$

{\bf (ii) ($\Rightarrow$)} $Y\subseteq \cup_{A\in Y/R}A$

{\bf (ii) ($\Leftarrow$)} $\cup_{A\in Y/R}A \subseteq Y$


\end{itemize}

\section{Preference Relation}

Consider a set $X$ of alternatives and binary relation $\prec$ on $X$. We interpret $x\prec y$ as ``$x$ is less preferred than $y$''. We define the indifference as the absence of strict preferences

\begin{equation*}
x\sim y \tiff (\tnot x\prec y, \tnot y\prec x)
\end{equation*}

\noindent {\bf Definition}

\begin{equation*}
x\precsim y\iff x\prec y \tor x\sim y
\end{equation*}

\noindent {\bf Theorem 1} {\it Suppose that $\prec$ on $X$ is a weak order. Then}

\begin{enumerate}
\item Exactly one of $x\prec y,x\sim y, y\prec x$ holds,

{\bf Proof} By definition we know that if $\sim$ where the case, then neither $x\prec y$ nor $y\prec x$ would hold.
At the same time, by its contrapositive, $\sim$ won't hold if either $x\prec y$ or $y\prec x$ holds, but, as $\prec$ is a weak order then we know that it is asymmetric, thus $x\prec y\implies \tnot y\prec x$ and vice versa. Therefore, only one of these can hold. $\Box$

\item $\prec$ is transitive,

{\bf Proof} If transitive, then there must exists $x,y,z\in X$ s.t. $x\prec y,y\prec z\implies x\prec z$. Now, assume not, then $\tnot x\prec z$. $\prec$ asymmetric implies that $\tnot z\prec y$, then, by negatively transitive we have $\tnot x\prec z,\tnot z\prec y\implies \tnot x\prec y$, which is a contradiction. $\Box$

\item $\sim$ is an equivalence,

{\bf Proof} We need to show that $\sim$ is reflexive, symmetric and transitive.

{\bf (symmetric)} Let $x,y\in X$ s.t. $\tnot x\prec y$ and $\tnot y\prec x$, then $x\sim y$. Now, by changing places, $\tnot y\prec x$ and $\tnot x\prec y$, then $y\sim x$. Thereby $\sim$ is symmetric. $\Box$

{\bf (transitive)} Assume $z\in X$ s.t. $x\sim y \tand y\sim z$. Now, by definition $(\tnot x\prec y,\tnot y\prec x)$ and $(\tnot y\prec z,\tnot z\prec y)$. By negatively transitivity of $\prec$ we have $\tnot x\prec z\tand \tnot z\prec x$, thus, by definition of $\sim$, $x\sim z$. $\Box$

{\bf (reflexive)} Let $x,y,z\in X$ s.t. $x\sim y$ and $x\sim z$. Then, by symmetry of $\sim$ we have $z\sim x$. Finally, $\sim$ transitive implies $x\sim x$. $\Box$

\item $(x\prec y,y\sim z)\implies x\prec z$ and $(x\sim y,y\prec z)\implies x\prec z$,

{\bf Proof}

{\bf ($(x\prec y,y\sim z)\implies x\prec z$)} By definition of $\sim$ we have $y\sim z\implies (\tnot y\prec z,\tnot z\prec y)$, now, since $x\prec y$, then asymmetry of $\prec$ implies  $\tnot y\prec x$. Furthermore, by negatively transitivity, we have $\tnot z\prec y,\not y\prec z\implies \tnot z\prec x$. Thereby, by the contrapositive of asymmetry we have $\tnot z\prec x\implies x\prec z$. $\Box$

{\bf ($(x\sim y,y\prec z)\implies x\prec z$)} By definition of $\sim$ we have $(\tnot x\prec y,\tnot y\prec x)$. Asymmetry implies that $\tnot z\prec y$, and negative transitivity that $\tnot z\prec y,\tnot y\prec x\implies \tnot z\prec x$, which by asymmetry implies $x\prec z$. $\Box$



\item $\precsim$ is transitive and connected.

{\bf Proof}

({\bf $\precsim$ transitive}) Let $x,y,z\in X$ s.t. $x\precsim y$ and $y\precsim z$, then, by definition of $\precsim$ we have $(x\prec y\tor x\sim y)$ and $(y\prec z\tor y\sim z)$, then we have 4 cases:

\begin{itemize}
\item $x\prec y\tand y\prec z$, which by transitivity of $\prec$ implies $x\prec z$,
\item $x\prec y\tand y\sim z$, which by proof implies $x\prec z$,
\item $x\sim y\tand y\prec z$, which by proof implies $x\prec z$
\item $x\sim y\tand y\sim z$, which by transitivity of $\sim$ implies $x\sim z$
\end{itemize}

This way, we have $(x\prec z\tor x\sim z)\implies x\precsim z$. $\Box$

({\bf $\precsim$ connected}) If connected, then either $x\precsim y$ or $y\precsim x$, which by definition implies $(x\prec y\tor y\sim x)\tor(y\prec x\tor y\sim x)$. Now, by symmetry of $\sim$, we have $x\prec y\tor y\prec x\tor x\sim y$, which was proved on theorem 1.1. $\Box$

\item Let $\prec'$ on $X/\sim$ defined by

\begin{equation*}
a\prec' b \iff x\prec y\text{ for some $x\in a$ and $y\in b$}
\end{equation*}

Then $\prec'$ on $X/\sim$ is a strict order.

{\bf Proof} Need to show that $\prec'$ on $X/\sim$ is weakly connected, weak order (asymmetric and negatively transitive).

{\bf (weakly connected)} 

{\bf (asymmetric)} Assume $a\prec' b$, then, for some $x\in a, y\in b$ we have $x\prec y$; furthermore, as $\forall x'\in a, x\sim x'$ and $\forall y'\in b, y\sim y'$, by transitivity of $\sim$ we have $x'\prec y'$. Now, as $\prec$ is asymmetric, we know that $\tnot y'\prec x'$. Finally, by contrapositive of $\prec'$, $\tnot y'\prec' x\implies \tnot b\prec' a$. $\Box$

{\bf (negatively transitive)} Let $\tnot a\prec' b$ and$\tnot b\prec' c$. Now, for some $x\in a, y\in b\tand z\in c$ we have the following: $\tnot x\prec y\tand\tnot y\prec z$, then by $\prec$ negatively transitive we have not $x\prec z$. Furthermore, as $\forall x'\in a, x'\sim x$ and $\forall y'\in c, z\sim z'$, and transitivity of $\sim$, we have not $x'\prec z'$, therefore, by the contrapositive of $\prec'$, not $x'\prec z'\implies\tnot a\prec' c$. $\Box$
\end{enumerate}

\section{An Order Preserving Utility Function}

\noindent {\bf Theorem 2.} If $\prec$ on $X$ is a weak order and $X/\sim$ is countable then there exists a real-valued function $u$ on $X$ s.t. for all $x,y\in X$,

\begin{equation*}
x\prec y\iff u(x)<u(y)
\end{equation*}

\noindent {\bf Definition:} The utility function $u$ is called a utility function of $\prec$.

\noindent {\bf Remark:} For any strictly increasing function $f$, if $u$ is utility function of $\prec$, then $f(u)$ is also a utility function of $\prec$.

\noindent {\bf Remark:} If $\prec$ on $X$ has a utility function $u$, then $\prec$ is a weak order.

\section{Preference as Strict Partial Order\label{sec:strictpartialorder}}

\noindent{\bf Definition} A briefly relation $R$ on a set $Y$ is a strict partial order iff it is reflexive and transitive.

\noindent{\bf Remark:} Remember that irreflexivity and negatively transitivity of $\prec$ implies transitivity of $\sim$.

\noindent{\bf Remark:} The irreflexivity and the transitivity of $\prec$ does not imply the transitivity of $\sim$.

\noindent{\bf Remark:} $x\approx y\implies x\sim y$

\noindent{\bf Definition:}

\begin{equation*}
x\approx y\iff (x\sim z\iff y\sim z\tforall z\in X)
\end{equation*}

\noindent{\bf Theorem 3.} Suppose $\prec$ on $X$ is a strict partial order, being irreflexive and transitive. Then

\begin{enumerate}
\item exactly one of $x\prec y,y\prec x,x\approx y, (x\sim y,\tnot x\approx y)$ holds for each $x,y\in X$,
\item $\approx$ is an equivalence,
\item $(x\prec y,y\approx z)\implies x\prec z$ and $(x\approx y,y\sim z)\implies x\prec z$,
\item Let $\prec^*$ defined on $X/\approx$ by
\begin{equation*}
a\prec^* b\iff x\prec y\text{ for some $x\in a$ and $y\in b$}
\end{equation*}

Then, $\prec^*$ on $X/\approx$ is a strict partial order.
\end{enumerate}

\section{Zorn's Lemma and Szpilrajn's Extension Theorem}

\noindent{\bf Theorem 4.} If $\prec^*$ is a strict partial order on a set $Y$, then there is a strict order $\prec^\circ$ on $Y$ that includes $\prec^*$, so that

\begin{equation*}
x\prec^* y\implies x\prec^\circ y
\end{equation*}

\noindent {\bf Zorn's Lemma:} Suppose that (i) $P$ on $Y$ is a strict partial order and (ii) for any subset $Z$ of $Y$ on which $P$ is a strict order, there is a $y\in Y$ s.t.

\begin{equation*}
zPy\tor z=y\tforall z\in Z.
\end{equation*}

Then, there is a $y^*\in Y$ s.t. $y^*Px$ for no $x\in Y$.

\def\precs{\prec^\circ}

\noindent{\bf Remark} Suppose $\prec^\circ$ is a strict partial order. If $\prec^\circ$ is weakly connected, then $\prec^\circ$ is a strict order.

\noindent{\bf Proof} We need to show that if $\prec^\circ$ is reflexive and transitive, by adding weakly connected we must obtain a strict order, this is weakly connected weak order (asymmetric and negatively transitive).

\noindent{\bf (asymmetric)} By contradiction, assume not, then for some $x,y\in X$ we should have $x\precs y$ and $y\precs x$, then, by transitivity of $\precs$ we get $x\precs x$, which contradicts weakly connected. $\Box$

\noindent{\bf (negatively transitive)} By contradiction assume not, then for some $x,y,z\in X$ we have $(\tnot x\precs y,\tnot y\precs z)\implies x\precs z$. Now, by asymmetry of $\precs$ we have $y\precs x$ and $z\precs y$; then, by weakly connected we know that $x\not=y\not=z$, and what's more, either $x\precs z$ or $z\precs x$, now, WLOG, assume the first, then, by transitivity of $\precs$ we have $y\precs z$, which is a contradiction. $\Box$

\section{Utility Representation for Strict Partial Order}

\noindent{\bf Theorem 5.} If $\prec$ on $X$ is a strict partial order and $X/\approx$ is countable then there is a real valued function $u$ on $X$ s.t. for all $x,y\in X$

\begin{enumerate}
\item $x\prec y\implies u(x)<u(y)$,
\item $x\approx y\implies u(x)=u(y)$.
\end{enumerate}

\end{document}
