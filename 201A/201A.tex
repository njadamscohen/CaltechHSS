\documentclass{article}

\usepackage{amsmath,amssymb,cancel}
\usepackage{hyperref}

\title{SS 201A: }
\author{George Vega Yon\and Nicholas Adams-Cohen}

% Some useful macros
\def\reals{\mathbb{R}}
\def\noR{\cancel{R}}

\begin{document}

\maketitle

\section*{Sep 29th, 2014}

\subsection*{Why study choice theory?}

The agent has a utility function and maximizes $\max_{x\in X}U(x)$, where $U$ is
a utiliy function and $X$ is the set of available actions.

\emph{What is the theorical foundation of this model?} is precisely what this
class is about.

Our basic tool is a preference relation $\prec$ or $R$. ``$x\prec y$'': agent 
prefers $y$ to $x$. More formally, the result we are going to obtain is: 
``$\prec$'' satisfies  axioms 1,...,$n$ iff there exists a function 
$U:X\to\reals$ s.t. $U(x)<U(y) \iff x\prec y$.

\begin{align*}
l&=(x_1;p;x_2;1-p) \\
r&=(y_1;q;y_2;1-q)
\end{align*}

$l\prec r \iff pU(x_1) + (1-p)U(x_2) < qU(y_1) + (1-q)U(y_2)$: von 
Neumann-Morgenstenn Expected utility theory.

\subsection*{Representation of Weak Order}

Assuming $X$ is a countable set:

\begin{itemize}
\item $\prec$ is \underline{weak order} iff $\exists U:X\to\reals$ s.t. $x\prec y
\iff U(x)<U(y)$.

\item \underline{Binary relation} on a set $Y$ is a set of ordered pairs $(x,y)$
with $x,y\in Y$.

\item \underline{Notation}:

\begin{align*}
R&:\text{Binary relation} \\
\prec&:\text{Preference relation}\\\\
xRy&: \text{ means }(x,y) \in R \\
x\noR y&:\text{ means }(x,y)\notin R
\end{align*}

\item \underline{Remark} if $R$ is a binary relation on $Y$, exactly one of the
following is true:
\begin{enumerate}
\item $xRy \wedge yRx$ or 
\item $xRy \wedge y\noR x$ or 
\item $x\noR y \wedge yRx$ or 
\item $x\noR y \wedge y\noR x$

\end{enumerate}
\end{itemize}

\noindent {\bf Example} $Y$ is the set of all living people, and $R_1:$``is shorter
than'' ($<$).

\noindent {\bf Definition} A binray relation $R$ on $Y$ is:
\begin{enumerate}
\item Reflexive if $xRx \forall x\in Y$.
\item Irriflexive if $x\cancel{R}x \forall x\in Y$.
\item Symmetric if $[xRy \vee yRx] \forall x,y\in Y$.
\item Asymmetric if $[xRy\implies x\noR y \forall x,y\in Y]$.
\item Antisymmetric if $[(xRy \wedge yRx)\implies x=y] \forall x,y\in Y$.
\item Negatively transitive if $[(x\noR y \wedge y\noR z)\implies x\noR z] \forall
x,y,z\in Y$.
\item Connected (complete) if $[xRy \vee yRx] \forall x,y\in Y$.
\item Weakly connected if $x\not=y\implies xRy \vee yRx\forall x,y\in Y$.
\end{enumerate}

\noindent {\bf Exercices} Show the following statements:
\begin{enumerate}
\item Asymetric binary relation is irriflexive.

\noindent {\bf Proof} Assume a relation which is not irreflexive, then $(x,x)\in
R \implies (xRx\implies x\noR x)$, which is a contradiction.

\item An irreflexive and transitive binray relation is asymmetric.

\noindent {\bf Proof} If $R$ were not asymetric it would be true that $xRy\implies yRx$, then, by
transitivity, $xRx$, this is, reflexivity, which is a contradiction.

\item A binary relation $R$ is neg. trans. iff $\forall x,y,z\in Y (xRy)\implies
(xRz \vee zRy)$.

\noindent {\bf Proof} By the conterpositive of $x\noR y \wedge y\noR z\implies x\noR z$, we know that
$xRz\implies xRy \vee yRz$. At the same time the counterpositive of $xRy\implies
xRz \vee zRy$ implies that $x\noR z \wedge z\noR y \implies x\noR y$.

\item Find all the properties of $R_1$ (shorter than).

\noindent {\bf Sol}
\begin{enumerate}
\item It can't be reflexive, ($A$ can't be taller than itself), then
\item It is irriflexive.
\item $R_1$ is a strict preference relation, thus it can't be symmetric, then
\item It is anyssmetric.
\item No
\item It is negatively transitive.
\item Irriflexibility implies that is not connected, furthermore,
\item It is weakly connected.
\end{enumerate}
\end{enumerate}

\end{document}
