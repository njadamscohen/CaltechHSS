\documentclass[10pt,a4paper,oneside]{article}
\usepackage{amssymb,amsmath,enumitem,fancyhdr,cancel,parskip,mathrsfs,amsthm,fullpage}
\begin{document}
\author{}
\title{SS 201a Notes}
\date{}
\maketitle

\newtheorem{Thm}{Theorem}
%Section 1
\section{Binary Relations}
{\bf Definition:} A binary relation on a set $Y$ is a set of ordered pairs $(x,y)$ with $x \in Y$ and $y \in Y$.\\ \\
{\bf Notation:} We often use $R$ to denote a binary relation. We denote $(x,y) \in R$ by $xRy$.\\ \\
{\bf Remark:} For any binary relation $R$ on $Y$ and if $x,y \in Y$, exactly one of the following is true.
	\begin{itemize}
		\item $(xRy, yRx)$
		\item $(xRy,$ not $yRx)$
		\item (not $xRy, yRX)$
		\item (not $xRy$, not $yRx)$
	\end{itemize}
{\bf Question} Why?\\ \\
{\bf Definition:} A binary relation $R$ on $Y$ is 
	\begin{itemize}
		\item reflexive if $xRx$ for every $x \in Y$.
		\item irreflexive if not $xRx$ for every $x \in Y$.
		\item symmetric if $xRy \Rightarrow yRx$ for every $x,y \in Y$.
		\item asymmetric if $xRy \Rightarrow$ not $yRx$ for every $x,y \in Y$ ($\Rightarrow R$ is irreflexive).
		\item antisymmetric if $(xRy,$ $yRx) \Rightarrow x=y$ for every $x,y \in Y$.
		\item transitive if $(xRy$, $yRz) \Rightarrow xRz$ for very $x,y,z \in Y$.
		\item negatively transitive if (not $xRy$, not $yRz) \Rightarrow $ not $xRz$ for every $x,y,z\in Y$.
		\item connected (or complete) if $xRy$ or $yRx$ for every $x,y \in Y$ ($\Rightarrow R$ is reflexive).
		\item weakly connected if $x \neq y \Rightarrow (xRy$ or $yRx)$ for every $x,y \in Y$.
	\end{itemize}
{\bf Remark:}
	\begin{itemize}
		\item An asymmetric binary relation is irreflexive.
		\item An irreflexive and transitive binary relation is asymmetric.
		\item A binary relation $R$ is negatively transitive if and only if for all $x,y,z \in Y$
	\end{itemize}
$$xRy \Rightarrow (xRz \text{ or }zRy).$$
%Section 2
\section{Weak Order}
{\bf Definition:} A binary relation $R$ on a set $Y$ is
	\begin{itemize}
		\item a weak order iff $R$ is asymmetric and negatively transtiive.
		\item a strict order iff $R$ is weakly connected weak order.
		\item an equivalence iff $R$ is reflexive, symmetric, and transitive.
	\end{itemize}
{\bf Definition:} Let $R$ be an equivalence class on $Y$. The set $$R(x)=\{y \in Y |yRx\}$$
	\hspace{12ex} is called the equivalence class generated by $x$.\\ \\
{\bf Remark:} $R(x)=R(y)$ if and only if $xRy$.\\ \\
{\bf Remark:} Any equivalence classes are either identical or disjoint.\\ \\
{\bf Notation:} We denote $\{R(x)\}_{x \in Y}$ by $Y/R$.\\ \\
{\bf Remark:} $Y/R$ is a partition of $Y$. That is
	\begin{enumerate}[label=(\roman*)]
	\item for any $A, A' \in Y/R$, if $A \neq A'$, then $A \cap A'=\phi$
	\item $Y=\bigcup_{A \in Y/R} A$.
	\end{enumerate}
%Section 3
\section{Preference Relation} 
\hspace{6pt} Consider a set $X$ of alternatives and a binary relation $\prec$ on $X$. We interpret $x \prec y$ as ``$x$ is less preferred than $y$.''\\
We define indifference as the absence of strict preferences
	\begin{equation*}
	x \sim y \text{ if and only if (not } x \prec y\text{, not } y \prec x).
	\end{equation*}
{\bf Interpretation:} Indifference might arise in several ways.
\begin{itemize}
	\item First, an individual might truly feel $x$ and $y$ are indifferent.
	\item Second, he may be uncertain as to his preference.
	\item Third, he may think $x$ and $y$ are incomparable.
\end{itemize}
{\bf Question:} Let $X$ be the set of all kinds of foods. Which property does your preference relation satisfy?\\ \\
{\bf Question:} Give an example in which your preference violates negative transitivity. \\ \\
{\bf Definition:} 
	$$x \precsim y \Leftrightarrow x \prec y \text{ or } x \sim y $$
\begin{Thm} Suppose that $\prec$ on $X$ is a weak order. Then
	\begin{enumerate}[label=\arabic*.]
		\item exactly one of $x\prec y$, $x \sim y$, $y \prec x$ holds,
		\item $\prec$ is transitive,
		\item $\sim$ is an equivalence,
		\item $(x\prec y, y \sim z ) \Rightarrow x \prec z$ and $(x \sim y, y \prec z ) \Rightarrow x \prec z$,
		\item $\precsim$ is transitive and connected.
		\item Let $\prec'$ on $X/\sim$ defined by $$a \prec b \Leftrightarrow x \prec y \text{ for some } x\in a \text{ and }y \in b.$$ Then $\prec'$ on $X/\sim$ is a strict order.
	\end{enumerate}
\end{Thm}
%Section 4
\section{An Order Preserving Utility Function}
\begin{Thm}
	If $\prec$ on X is a weak order and $X/\sim$ is countable, then there exists a real-valued function u on X such that for all x,y $\in$ X, $$x \prec y \Leftrightarrow u(x) < u(y).$$
	\end{Thm}
{\bf Definition:} The utility function $u$ is called a \emph{utility function} of $\prec$.\\ \\
{\bf Remark:} For any monotonic increasing function $f$, if $u$ is a utility function of $\prec$, then $f(u)$ is also a utility function of $\prec$.\\ \\
{\bf Remark:} If $\prec$ on $X$ has a utility function $u$, then $\prec$ is a weak order.\\ \\
{\bf Remark:} If $\prec$ is a weak order on $Y$ and $Z \subset Y$ is finite, then there exists $z^* \in Z$ so that $z^* \prec z'$ for all $z' \in Z$. ($z' \neq z^*$)\\ \\
{\bf Definition:} $x_1 \prec x_2 \prec \cdots \prec x_n$ for some $n \Rightarrow x_1 \neq x_n$ (acyclicity)\\ \\
{\bf Remark:} If a binary relation $R$ is a weak order on $Y$, it is acyclic. 
%Section 5
\section{Preference as a Strict Partial Order}
In this section, we look at the case where indifference is not assumed to be transitive.\\ \\
{\bf Definition:} A binary relation $R$ on a set $Y$ is a strict partial order if and only if it is irreflexive and transitive.\\ \\
{\bf Remark:} Irreflexivity and transitivity imply asymmetry of $R$. So under transitivity, asymmetry and irreflexitivity are equivalent.\\ \\
{\bf Remark:} Remember that $\prec$ being negative transitive implies that $\sim$ is transitive.\\ \\
{\bf Remark:} Irreflexivity and transitivity do not imply transitivity of $\sim$.\\ \\
{\bf Remark:} If $\prec$ is a strict partial order, then $\sim$ is symmetric and reflexive.\\ \\
We introduce a new concept:\\ \\
{\bf Definition:} $$x \approx y \Leftrightarrow (x \sim z \Leftrightarrow y \sim z \text{ for all } z \in X)$$\\
{\bf Remark:} If $x \approx y$, then $x \sim y$.\\
\begin{Thm} Suppose $\prec$ on X is a strict partial order, being irreflexive and transitive. Then
	\begin{enumerate}[label=\arabic*.]
		\item exactly one of $x\prec y$, $y\prec x$, $x \approx y$, $(x \sim y$, not $x \approx y)$ holds for each $x, y \in X$,
		\item $\approx$ is an equivalence,
		\item $(x \prec y$, $ y \approx z) \Rightarrow x \prec z$ and $(x \approx y$, $y  \prec z ) \Rightarrow x \prec z$,
		\item Let $\prec^*$ be defined on $X/\approx$ by $$a \prec^* b \Leftrightarrow x \prec y \text{ for some } x \in a \text { and } y \in b$$ Then, $\prec^*$ on $X/\approx$ is a strict partial order.
	\end{enumerate}
\end{Thm}
%Section 6
\section{Zorn's Lemma and Szpilrajn's Extension Theorem}
{\bf Zorn's Lemma:} Suppose that 
\begin{enumerate}[label=(\roman*),leftmargin=*]
	\item $P$ on $Y$ is a strict partial order and 
	\item for any subset $Z$ of $Y$ on which $P$ is a strict order, there is a $y \in Y$ such that $$zPy \text{ or } z=y\text{ for all }z \in Z.$$ 
\end{enumerate}
Then, there is a $y^* \in Y$ such that $y^* P x$ for no $x \in Y$.\\ \\
{\bf Remark:} If a strict partial order $P$ is weakly connected, then $P$ is a strict order.\\
\begin{Thm}
	If $\prec^*$ is a strict partial order on a set $Y$, then there is a strict order $\prec^0$ on Y that includes $\prec^*$, so that $$ x \prec^* y \Rightarrow x \prec^0 y.$$
\end{Thm}
%Section 7
\section{Utility Representation for Strict Partial Order}
\begin{Thm}
	If $\prec$ on X is a strict partial order and $X/\approx$ is countable, then there is a real valued functino u on X such that for all x, y $\in $X
	\begin{enumerate}[label=\arabic*.]
	\item $x \prec y \Rightarrow u(x)<u(y).$
	\item $x \approx y \Leftrightarrow u(x)=u(y).$
	\end{enumerate}
\end{Thm}

\end{document}